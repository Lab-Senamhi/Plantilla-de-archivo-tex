\documentclass[a4paper, 11pt]{article}
\usepackage{fullpage} % changes the margin

\usepackage[spanish,es-tabla,es-lcroman]{babel}
\usepackage[utf8]{inputenc}
\usepackage{graphicx}

\usepackage{verbatim} %para el código

\usepackage{amsmath} % para la matemática
\usepackage{amsfonts}
\usepackage{amssymb}

\begin{document}
%Header-Make sure you update this information!!!!
%Encabezado - Por favor actualiza esta informacion
\noindent
\large\textbf{Reporte Lab X } \hfill \textbf{Nombre Apellidos} \\
\normalsize IMTR 3013 - Visión Artificial \hfill Código: T000XXXX \\
Prof. A. Marrugo \hfill Fecha: XX/XX/XXXX \\
Dept. Mecánica y Mecatrónica \hfill Universidad Tecnológica de Bolívar veamos que sucede...........

\section*{Introducción}
Aquí va la descripción general del problema, motivación o la temática de la que trató la práctica de laboratorio. Un ejemplo de una cita~\cite[p.219]{Robotics}. Aquí hay otra cita~\cite{Flueck}.

\section*{Marco teórico}
Se debe incluir una breve descripción de la teoría detrás de la práctica. Por ejemplo, puede citarse apartados del libro guía, o alguna lectura que complementaria.

Si se debe especificar alguna ecuación, se hace de la siguiente manera:

\begin{align}
y = mx+b
\label{eq:linea}
\end{align}

La ecuación~\eqref{eq:linea} describe una línea recta.

\section*{Implementación}
Aquí van los detalles del código y cualquier dato interesante durante la implementación y/o ejecución del código.

Se desarrolló la práctica de la siguiente manera:
\begin{enumerate}
	\item Paso 1
	\item Paso 2
\end{enumerate}	

Aquí va un ejemplo de código:

\begin{verbatim}
% Filtro Promedio
filprom = (1/9)*[1 1 1;1 1 1;1 1 1]
dat = imread('eight.tif');
datruid = imnoise(dat,'salt & pepper',0.02);
\end{verbatim}

\section*{Resultados}
Aquí van los resultados. Puede incluir figuras, gráficos, tablas, etc. En la figura~\ref{fig:01} se muestra la imagen de calibración.


\begin{figure}
	\centering
		\includegraphics[width=0.5\textwidth]{imagen01.png}
	\caption{Imagen de prueba.}
	\label{fig:01}
\end{figure}

También se debe incluir las respuestas/análisis a preguntas propuestas en la guía.

\section*{Conclusiones}
Aquí van las conclusiones y cualquier otra observación que se quiera destacar.

\section*{Archivos adjuntos}
% Modifica esto segun tus necesidades
Imágenes: (im1.png), script01.m, funcion01.m

\begin{thebibliography}{9}
\bibitem{Robotics} Fred G. Martin \emph{Robotics Explorations: A Hands-On Introduction to Engineering}. New Jersey: Prentice Hall.
\bibitem{Flueck}  Flueck, Alexander J. 2005. \emph{ECE 100}[online]. Chicago: Illinois Institute of Technology, Electrical and Computer Engineering Department, 2005 [cited 30
August 2005]. Available from World Wide Web: (http://www.ece.iit.edu/~flueck/ece100).
\end{thebibliography}

\end{document}
